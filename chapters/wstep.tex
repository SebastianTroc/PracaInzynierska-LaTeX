\chapter{Wstęp}
\section{Idea pracy dyplomowej}

Niniejsza praca opisuje projekt i wykonanie mobilnego systemu komunikacji przeznaczonego dla Ośrodków Szkolenia Kierowców roboczo nazwanego ,,OSK-Helper''. Jego architekturę, część wizualną interfejsów, sposób instalacji i~uruchomienia, zalecane przypadki użycia oraz opracowane lub wykorzystane algorytmy.


\section{Założenia projektu}

Głównym założeniem projektu było stworzenie prostej aplikacji mobilnej, która będzie przyjazna dla użytkowników, a jednocześnie nadal będzie spełniała swoje zadanie - ułatwiała komunikacji pomiędzy instruktorami Ośrodków Szkolenia Kierowców w celu zmniejszenia kolizji zajętości placów manewrowych.

Kolejnym założeniem było wykorzystanie języka JavaScript w każdej płaszczyźnie aplikacji, zarówno klienta, jak i serwera oraz użycie nierelacyjnej bazy danych, która będzie przechowywała dane o strukturze zbliżonej do struktury obiektów tego języka.


\section{Cel pracy}

Celem pracy jest projekt i realizacja systemu mającego za zadanie ułatwić pracę instruktorów Prawa Jazdy poprzez zautomatyzowanie wymiany informacji dotyczącej zajętości placów manewrowych, z których mogą korzystać.


\section{Zakres pracy}

Projekt praktyczny, stworzony dla celów niniejszej pracy jest znaczną częścią trudu włożonego w jej powstanie i w jego skład wchodzi:

\begin{itemize}
	\item aplikacja mobilna dla systemu Android
	\item aplikacja serwerowa będąca zapleczem dla mobilnego systemu.
\end{itemize}